\section{Zusammenfassung}
\label{sec:zusammenfassung}
Das Ziel dieser Diplomarbeit war es, ein ferngesteuertes Modellauto so umzubauen, dass diverse Messdaten, darunter die Beschleunigung entlang der x- und y- Achse des Autos, die Lage im dreidimensionalen Raum, die Drehzahlen der vier Räder und die \ac{GPS}-Koordinaten, für die Analyse aufgezeichnet und für den Nutzer zugänglich gemacht werden können. Als Format der Datenaufzeichnung wurde das \ac{CSV}-Format gewählt, um die Kompatibilität mit Tabellenkalkulationsprogrammen zu gewährleisten.\\
Als Sensor für die Aufzeichnung der Beschleunigung und der Lage im Raum wurde die \ac{IMU} MPU9250 gewählt, da diese günstig ist und trotzdem alle benötigten Werte in einem Paket liefert. Um die Lage zu bestimmen wurde Sensorfusion verwendet, um zuverlässige Werte zu erhalten. Aufgrund der zu Beginn nicht bedachten Vibrationen des Motors und des Getriebes sind die Werte der \ac{IMU} jedoch aktuell unbrauchbar.\\
Für die Aufzeichnung der Drehzahlen der vier Räder wurden zwei verschiedene Sensortechniken gewählt, für die Vorderräder werden aufgrund der Beständigkeit gegen Verschmutzungen und Sonneneinstrahlung Hall-Effekt-Sensoren, für die Hinterräder werden Infrarot-Gabellichtschranken verwendet. Da beide Sensoren nur 0 bzw. 1 als Ausgangssignal liefern, wurden beide Sensortypen programmatisch gleich eingebunden. In beiden Fällen wird die Zeit zwischen den Sensorimpulsen verwendet, um die Drehzahl zu berechnen. Zusätzlich dazu wird die Geschwindigkeit des Autos mithilfe der Drehzahl und des Raddurchmessers errechnet.\\
Für die Aufzeichnung der \ac{GPS}-Koordinaten wurde das \ac{GPS}-Modul \glqq GY-NEO6MV2\grqq\ gewählt. Dieses Modul liefert die Koordinaten in Form des \ac{NMEA}-Protokolls. Mithilfe einer Python-Bibliothek wurden aus diesem Protokoll die tatsächlichen \ac{GPS}-Koordinaten entnommen.\\
Für die Datenverarbeitung wurde der Raspberry Pi Zero 2 W ausgewählt, da dieser alle benötigten Schnittstellen unterstützt, genügend Rechenleistung hat und ein gutes Preis-Leistungs-Verhältnis aufweist. Das Programm zur Datenaufzeichnung wurde in Python geschrieben.\\
Zusätzlich wurde eine Desktopanwendung mithilfe von Python und Qt geschrieben, um eine einfache Auswertung der Daten zu ermöglichen. Die Optionen zur Veranschaulichung sind die \glqq Table View\grqq , die \glqq Plot View\grqq\ und die \glqq Map View\grqq .\\
Um die Veranschaulichung aller realen Teile dieser Diplomarbeit zu verbessern, wurde das gesamte Modellauto sowie die von den Diplomanden eingebaute Elektronik \ac{3D}-modelliert beziehungsweise konstruiert. \\
\\
Diese Diplomarbeit wurde im Zuge der schriftlichen Reife- und Diplomprüfung an der HTBL Kapfenberg durchgeführt sowie von dieser finanziert.