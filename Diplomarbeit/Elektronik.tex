\section{Elektronischer Aufbau}
\label{sec:Elektronik}
Um ein gesamtes funktionsfähiges Mess- und Verarbeitungssystem realisieren zu können, sind mehrere Schritte, hinsichtlich des mechanischen und elektronischen Aufbaus notwendig. Dazu zählen unter anderem die Wahl der Hardwareplattform, die Auslegung eines Spannungsversorgungssystems und Möglichkeiten den Zustand des Programmes erkennen und ändern zu können. Außerdem ist ein robustes Gehäuse zur Einhausung dieser Komponenten sowie mehrerer Sensoren zu konstruieren und fertigen. Zusätzlich sind externe, im Gehäuse nicht verbaute Sensoren, über eine zuverlässige Verbindung in das Elektroniksystem einzubinden.

\subsection{Spannungsversorgung}
\label{subsec:elekSupply}
Da das Auto, wie in Sektion \ref{sec:Auto} beschrieben, mit zwei 7.4\ac{V} Lithium Polymer Akkus betrieben wird, bietet sich die Möglichkeit an, diese auch für die Versorgung der Messelektronik zu verwenden. Um die Akkus gleichmäßig zu belasten werden beide in Serie geschalteten Akkus für diese Spannungsversorgung verwendet. Damit stehen 14.8\ac{V} zur Verfügung, welche mit einem DC-DC Stepdown-Converter auf 5\ac{V}, was der Versorgungsspannung des RaspberryPi entspricht, herabgewandelt werden. Der DC-DC-Converter weist am Eingang zwei Schraubklemmen sowie eine Hohlbuchse auf. Die Leitungen der Akkus werden an den Schraubklemmen am Eingang des Converters angeschlossen. Ein Kippschalter, der die Plus-Leitung unterbrechen kann, ermöglicht das manuelle Aus- und Einschalten der gesamten Messelektronik. Zusätzlich wird ein Voltmeter mit Siebensegmentanzeige an der Hohlstecker Verbindung am Eingang angeschlossen, welches die Akkuspannung anzeigt. Der Stepdown-Wandler weist am Ausgang ebenfalls zwei Schraubklemmen und zusätzlich eine \ac{USB}-A-Buchse auf. Da der RaspberryPi als Versorgungsanschluss eine Micro-\ac{USB}-Buchse verbaut hat, wird ein USB-A zu Micro-\ac{USB} Kabel gekürzt und als Versorgungsleitung verwendet. Die Spannungsversorgung der einzelnen Sensoren erfolgt über die 3.3\ac{V}, 5\ac{V} und GND Pins des RaspberryPi.

\subsection{Gehäuse}
\label{subsec:elekCasing}
Eine robuste, vielseitige Einhausung der Messelektronik ist essentiell, um diese stabil montieren sowie von äußeren Einflüssen wie Staub und Sand schützen zu können. Die Sitzkonstruktion des Autos wurde entfernt, wodurch Raum für ein Gehäuse freigelegt wurde. Als Basis dient eine einfache Plattform, welche in das Auto nieder geschraubt wird, so wie die Sitzkonstruktion ebenfalls montiert war. Direkt darunter auf der Bodenplatte des Autos, befinden sich die Akkus. Auf diese Plattform werden mehrere Seitenwände aufgeschraubt.\\
Sämtliche Schraubverbindungen im und am Gehäuse werden mittels M3 Gewindeeinsätzen ermöglicht. Die \ac{FDM}-3D-gedruckten Gehäusekomponenten weisen Bohrungen auf, in welche diese Gewindeeinsätze eingeschmolzen werden. Diese Lösung, Schraubverbindungen in 3D-Druck-Komponenten zu ermöglichen, ist sehr belastbar und einfach umzusetzen und wird deshalb auch häufig eingesetzt. \\
Die U-förmige Rück- und gleichzeitig hintere Seitenwand, weist in Bodennähe eine rechteckige Ausnehmung auf, welche für die Durchführung der Versorgungsleitungen sowie den Sensorleitungen der hinteren Drehzahlsensoren vorgesehen ist. Auf der rechten Seite dieser Wand ist der Kippschalter der Versorgungsleitung eingefasst. Links befindet sich die \ac{USB}-A-Buchse für die Datenübertragung mittels \ac{USB}-Datenträger, wie es in Sektion \ref{subsec:Datenübertragungsmodus} näher beschrieben wird.