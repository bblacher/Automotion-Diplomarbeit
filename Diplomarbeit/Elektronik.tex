\section{Elektronischer Aufbau}
\label{sec:Elektronik}

\subsection{Wahl der Hardware-Plattform}
\label{subsec:elekMikrocontroller}
An die Hardware-Plattform werden einige Anforderungen gestellt. Sie muss Möglichkeiten zur Kommunikation mit den Sensoren bieten, außerdem muss sie genug Rechenleistung aufbringen können, um alle Sensoren simultan auszulesen und diese Daten für die spätere Verwendung aufzuzeichnen. Es sollten außerdem Optionen vorhanden sein, die gesammelten Daten vom ferngesteuerten Auto zu exportieren, um sie auf einem anderen Gerät auszuwerten. Eine weitere essentielle Eigenschaft ist, in welchen Programmiersprachen die Hardware programmiert werden kann, da die Verfügbarkeit von Bibliotheken für die jeweilige Programmiersprache eine wichtige Rolle spielt. Andere Faktoren, die zwar nicht notwendig sind, aber von großem Nutzen sein können, sind \ac{WLAN}- und Bluetooth-Funktionalität. Die bekanntesten Hersteller von solchen Plattformen sind die Raspberry Pi Foundation und Arduino.  Beide Hersteller bieten sowohl größere und leistungsstärkere als auch kleinere, leistungsschwächere Optionen an. Die bekanntesten Optionen sind somit der Raspberry Pi 4 Model B, der Raspberry Pi Zero 2 W, der Arduino Uno und der Arduino Nano. 
\begin{table}[h]
\centering
\begin{tabular}{|c||c|c|c|c|} 
\hline
Plattform           & Arduino Uno                                                                                    & Arduino Nano                                                                                   & Raspi 4B                                                                                                                                  & Raspi Zero 2W                                                                                                                              \\ 
\hhline{|=::====|}
Prozessor           & ATmega328P                                                                                     & ATmega328                                                                                      & BCM2711                                                                                                                          & Cortex-A53                                                                                                                             \\ 
\hline
Schnittstellen      & \begin{tabular}[c]{@{}c@{}}UART\\SPI\\I$^2$C\\6 analoge Pins\\14 digitale Pins\end{tabular} & \begin{tabular}[c]{@{}c@{}}UART\\SPI\\I$^2$C\\8 analoge Pins\\12 digitale Pins\end{tabular} & \begin{tabular}[c]{@{}c@{}}UART\\SPI\\I$^2$C\\28 \ac{GPIO} pins\end{tabular}                                          & \begin{tabular}[c]{@{}c@{}}UART\\SPI\\I$^2$C\\28 \ac{GPIO} pins\end{tabular}                                           \\ 
\hline
\begin{tabular}[c]{@{}c@{}}Programmier-\\sprachen\end{tabular} & C, C++                                                                                         & C, C++                                                                                         & \begin{tabular}[c]{@{}c@{}}Python\\C, C++\\Scratch\end{tabular} & \begin{tabular}[c]{@{}c@{}}Python\\C, C++\\Scratch\end{tabular}  \\ 
\hline
\begin{tabular}[c]{@{}c@{}}Zusatz-\\funktionen\end{tabular}    & -                                                                                              & -                                                                                              & \ac{WLAN}, Bluetooth                                                                                                     & \ac{WLAN}, Bluetooth                                                                                                      \\ 
\hline
Preis               & 20\officialeuro                                                                 & 18\officialeuro                                                                 & ab \$35                                                                                                                   & \$15                                                                                                                       \\
\hline
\end{tabular}
\caption{Vergleich von verschiedenen Hardware-Plattformen}
\label{tab:myCcomparison}
\end{table}
\\
Da auf dem Raspberry Pi wie in Sektion \ref{subsec:tRasPiOS} beschrieben eine vollwertige Linux-Distribution läuft, können auf Einplatinencomputer der Raspberry Pi Foundation alle programmiersprachen verwendet werden, die unter Linux unterstützt sind. Die in der Tabelle aufgelisteten Sprachen sind die, die von Raspberry Pi OS standardmäßig unterstützt werden, ohne eigens Pakete installieren zu müssen.\\
Aufgrund der im Vergleich zum ATmega328 hohen Rechenleistung des Arm Cortex-A53, der Option, Einplatinencomputer der Raspberry Pi Foundation mit Python zu programmieren, der zusätzlichen \ac{WLAN}-Funktion und den niedrigen Preises wird der Raspberry Pi Zero 2 W verwendet.