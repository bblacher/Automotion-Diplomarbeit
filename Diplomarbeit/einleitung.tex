\section{Einleitung}
\label{sec:einleitung}
Diese Diplomarbeit handelt von der Ausstattung eines ferngesteuerten Autos mit Sensorik, finanziert wurde sie von der HTL Kapfenberg.\\
Das Ziel ist es, verschiedene Kenngrößen, die eine Analyse des Fahrverhaltens ermöglichen, zu messen. Darunter fallen die Beschleunigung entlang der x- und y-Achse, die Neigung um die x-, y- und z-Achse, die Drehzahl von allen vier Rädern sowie die Position. Diese Daten sind digital festzuhalten und anschließend auszuwerten. \\
Um dies zu erreichen werden passende Sensoren sowie ein Einplatinencomputer zur Datenverarbeitung ausgewählt. Zur Befestigung der gewählten Elektronik werden Halterungen mithilfe von \ac{CAD}-Software konstruiert und anschließend mit verschiedenen 3D-Druck-Verfahren gefertigt. Die Spannungsversorgung der Bauteile erfolgt über die vom RC-Auto verwendeten Akkus, wessen Spannung auf die notwendige herabgewandelt wird. Außerdem werden die Sensoren in einem gemeinsamen Python-Programm eingebunden um die Sammlung der Daten in einer einzelnen Datei (als .txt - mit Kommas getrennt) zu ermöglichen. Weiters kann mittels Knopfdruck zwischen zwei Modi gewechselt werden, welche mit dem Status einer \ac{LED} unterschieden werden können. Zum Einen steht das Aufzeichnen der Daten zur Verfügung, andererseits können die gesammelten Daten auf einen \ac{USB} Datenträger übertragen werden. \\
Die Auswertung der Daten wird mithilfe einer eigens geschriebenen Desktop-Anwendung vereinfacht, diese wird ebenfalls mithilfe von Python umgesetzt. Zusätzlich wird das Grafikoberflächen-toolkit Qt für das Darstellen der Fenster verwendet.